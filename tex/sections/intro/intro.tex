%! TEX root = ../../main.tex

\documentclass[main]{subfiles}

\begin{document}

\chapter{はじめに}
\section{研究背景}
これは\LaTeX のテンプレートワークスペースです.
\subsection{機能}
次のような機能を持っています.
\begin{description}
  \item[Latex] \LaTeX の執筆,コンパイルを行う.基本的な機能は\TeX Liveで提供される.
  \item[Latex Workshop] VSCodeエディタの拡張機能で,\LaTeX のコンパイルを行う.保存時やワンクリックでコンパイルが可能.
  \item[Text Lint] 日本語の文章を校正する.
  \item[PlantUML] UML図を作成する.
  \item[Graphviz(dot)] ダイアグラム図を作成する.
  \item[Draw.io] Draw.ioを扱う.
\end{description}

\subsection{構成}

\section{\LaTeX 基本機能}

\subsection{見出し}

{\let\clearpage\relax \chapter*{章}}

\section*{節}

\subsection*{小節}

\subsubsection*{小々節}

\paragraph*{段落}
\jalipsumiroha

\subparagraph*{小段落}
\jalipsumiroha


\subsection{字体}

\textbf{bold}

\textit{italic}

\texttt{type writer}

\textgt{gothic}


\subsection{箇条書き}

\subsubsection{箇条書き}
\begin{itemize}
  \item 箇条書き
  \item 箇条書き
  \item 箇条書き
\end{itemize}


\subsubsection{番号付き箇条書き}
\begin{enumerate}
  \item 番号付き箇条書き
  \item 番号付き箇条書き
  \item 番号付き箇条書き
\end{enumerate}


\subsubsection{説明付き箇条書き}
\begin{description}
  \item[説明付き箇条書き] 説明付き箇条書き
  \item[いろは] 説明付き箇条書き
  \item[にほへ] 説明付き箇条書き
\end{description}


\subsection{表}

\begin{table}[h]
  \caption{簡単な表の例}
  \label{tab:easy}
  \centering
  \begin{tabular}{lcc} \toprule
           & A & B \\ \midrule
    \LaTeX & 1 & 2 \\
    \TeX   & 3 & 4 \\ \bottomrule
  \end{tabular}
\end{table}

\begin{table}[!h]
  \centering
  \caption{複雑な表の例}
  \label{tab:comp}
  \begin{tabular*}{\textwidth}{@{\extracolsep{\fill}}lcccccc}
    \toprule
    & \multicolumn{3}{c}{\textbf{Paired Differences1}} & \multicolumn{3}{c}{\textbf{Paired Differences2}}\\
    \cmidrule(r){2-4} \cmidrule(l){5-7}
    & \thead{Statistic}
    & \thead{df}
    & \thead{Sig.}
    & \thead{Statistic}
    & \thead{df}
    & \thead{Sig.}    \\
    \midrule
    Difference & 44.20 & 14.36 & 4.54 & .957 & 10 & .746\\
    \bottomrule
  \end{tabular*}
\end{table}

\subsection{図}

\begin{figure}[h]
  \centering
  \includegraphics[width=0.5\textwidth]{img/latex.png}
  \caption{図の例}
  \label{fig:latex}
\end{figure}

\end{document}